\documentclass[11pt, a4paper]{article}

\usepackage[utf8]{inputenc}
\usepackage{amsfonts}
\usepackage{amsmath}
\usepackage{hyperref}
\usepackage{graphicx}
\usepackage[frenchb]{babel}
\usepackage[left=3cm, right=3cm]{geometry}
\usepackage{titling}
\usepackage{enumitem}

\newcommand{\subtitle}[1]{%
  \posttitle{%
    \par\end{center}
    \begin{center}\large#1\end{center}
    \vskip0.5em}%
}

\SetLabelAlign{parright}{\parbox[t]{\labelwidth}{\raggedleft#1}}
\setlist[description]{
  style=multiline,
  topsep=10pt,
  leftmargin=5cm,
  align=parright
}

% Headers & footers
\usepackage{fancyhdr}
\pagestyle{fancyplain}
\fancyhead[LE]{\fancyplain{}{}}
\fancyhead[CE]{\fancyplain{}{}}
\fancyhead[RE]{\fancyplain{}{\bfseries\leftmark}}
\fancyhead[LO]{\fancyplain{}{\bfseries\rightmark}}
\fancyhead[CO]{\fancyplain{}{}}
\fancyhead[RO]{\fancyplain{}{}}
\fancyfoot[LE]{\fancyplain{}{}}
\fancyfoot[CE]{\fancyplain{}{}}
\fancyfoot[RE]{\fancyplain{}{}}
\fancyfoot[LO]{\fancyplain{}{}}
\fancyfoot[CO]{\fancyplain{}{\bfseries\thepage}}
\fancyfoot[RO]{\fancyplain{}{}}
% \renewcommand{\footrulewidth}{0.4pt}

\begin{document}

\title{Dossier d'initialisation}
\subtitle{Dématérialisation d'un processus de paiement}
\author{
  COMETS Jean-Marie \\
  DELMARRE Adrian \\
  REYNOLDS Nicolas \\
  TURPIN Pierre
}
\date{\today}

\maketitle \newpage

\tableofcontents \newpage

\section{Contexte et objet du projet} % TODO
Établir une offre technico-commerciale de solution de ticket restaurant
numérique, proposant les points suivants :
\begin{itemize}
  \item introduire des caisses rapides en grande surface permettant à au client
    d'éviter une longue attente en prenant son repas de midi
  \item automatiser le paiement des tickets restaurants par l'entreprise ainsi
    que la distribution des tickets restaurants
\end{itemize}

Cette offre sera surtout fournie d'un business plan détaillé pour appuyer
l'intérêt de l'offre. \\

\section{Description des livrables}
\begin{description}
    \item[Dossier d'initialisation] ce document, fourni dès l'entame du projet.
    \item[Expression des besoins] description de la solution demandée et mise en
        rapport avec la solution envisagée.
    \item[Description des solutions] architecture applicative et technique, ainsi
        que benchmarking des solution de paiement et de leur interconnexion avec un
        SI.
    \item[Business plan] définition des objectifs, description de l'organisation
        existante et des moyens à mettre en oeuvre pour atteindre les objectifs
        fixés.
\end{description}

\section{Gestion des tâches} % TODO

\section{Organisation de l'équipe}
\begin{itemize}
    \item Jean-Marie Comets (chef de projet)
    \item Pierre Turpin (responsable qualité)
    \item Adrian Delmarre
    \item Nicolas Reynolds
\end{itemize}

\section{Analyse des risques}

\subsection{Facteurs de risque}

\begin{description}
    \item[Difficultés techniques] La spécificité du domaine d'étude demande une
        forte spécialisation des membres de l'équipe.
    \item[Degré d'intégration et taille du projet] Le projet est ambitieux et
        complexe. Son envergure laisse présager une architecture modulaire
        entraînant des dépendances et interactions fortes. Une forte rigueur
        organisationnelle sera nécessaire, et on peu envisager le recours à des
        méthodes de développement Agile (de type Scrum).
    \item[Instabilité de l'équipe] Une émulation mutuelle est nécessaire pour
        prévenir une éventuelle démotivation de l'équipe.
\end{description}

\subsubsection{Configuration organisationnelle}

\begin{description}
    \item[Travail collaboratif] Mise en place de normes, d'un guide de style,
        de chartes graphiques et d'outils collaboratifs.  Définition du
        workflow-type du cycle de production et de validation d'un livrable.
    \item[Echéances] Il est impératif de mettre en place un suivi régulier de
        la planification et d'un échéancier large. Le chef de projet et le
        responsable qualité doivent veiller régulièrement au respect des
        échéances et anticiper les suivantes.
\end{description}

\subsection{Risques liés au projet}

\begin{description}
    \item[Risques financiers] En l'occurrence, il s'agira principalement de
        dépassement de volume raisonnable de travail.  Il incombe au chef de
        projet de mener à bien une planification stricte structurée par un
        suivi régulier du temps de travail sur chaque tâche. Il sera nécessaire
        de prévoir des indicateurs significatifs de ces aspects sur les
        tableaux de bord de suivi de projet.
    \item[Risques humains] D'éventuels cas d'incompétence seront prévenus par
        l'allocation de créneaux de veille technologique, et le recours
        systématique à l'entraide.
    \item[Risques technologiques] Pour prévenir la perte de documents, un
        gestionnaire de version sera systématiquement utilisé.
\end{description}

\end{document}

% vim: ft=tex et sw=4 sts=4
