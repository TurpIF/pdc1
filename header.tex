\documentclass[10pt, a4paper]{article}

% On écrit en français
\usepackage[utf8]{inputenc}
\usepackage[frenchb]{babel}
\usepackage[T1]{fontenc}

% Packages nécessaires
\usepackage{graphicx}
\usepackage{hyperref}
\usepackage{eurosym}
\usepackage{wrapfig}
\usepackage{lscape}

% Diagrammes de séquences
\usepackage{tikz}
\usetikzlibrary{arrows, shadows}
\usepgflibrary{arrows}
\usepackage[underline=true, rounded corners=false]{pgf-umlsd}

% Document "fancy"
\usepackage{fancyhdr}
\pagestyle{fancy}
\fancyhf{}

% Gros en-têtes/pied de pages
\renewcommand{\headrulewidth}{2pt}
\renewcommand{\footrulewidth}{1pt}

% Police Helvetica <3
\usepackage{helvet}
\renewcommand*{\familydefault}{\sfdefault}

% Enlever les alinéas
\setlength{\parindent}{0pt}

% Sous titre de document
\usepackage{titling}
\newcommand{\subtitle}[1]{%
  \posttitle{%
    \par\end{center}
    \begin{center}\large#1\end{center}
    \vskip0.5em}%
}

% Nom du projet
\def \DocumentProject {Dématérialisation d'un processus de paiement}

% En tête complet de document
\newcommand{\Document}[1]{%
    \def \DocumentTitle {#1}

    \title{\DocumentTitle}
    \subtitle{\DocumentProject}
    \author{
        COMETS Jean-Marie \\
        DELMARRE Adrian \\
        REYNOLDS Nicolas \\
        TURPIN Pierre
    }
    \date{\today}

    \maketitle \newpage

    \tableofcontents \newpage
}

% Commande pour faire des CU (sans la description détaillée du processus)
\usepackage{array}
\newcommand{\CUBref}[6]{
  \textbf{#1}

  \textbf{Description} \\
  #2 \\

  \textbf{Acteur} \\
  #3 \\

  \textbf{Précondition} \\
  #4 \\

  \textbf{Conséquence} \\
  #5 \\

  \textbf{Exception} \\
  #6
  ~\\

%   \renewcommand{\arraystretch}{1.75}
%   \setlength{\tabcolsep}{0.5cm}
  % \begin{tabular}{rp{0.35\textwidth} lp{0.65\textwidth}}
  %   \multicolumn{2}{l}{\large{\textbf{#1}}} \\
  %   \textbf{Description} & \begin{minipage}[l]{0.65\textwidth}#2\end{minipage} \\
  %   \textbf{Acteur} & #3 \\
  %   \textbf{Précondition} & \begin{minipage}[l]{0.65\textwidth}#4\end{minipage} \\
  %   \textbf{Conséquence} & \begin{minipage}[l]{0.65\textwidth}#5\end{minipage} \\
  %   \textbf{Exception} & \multicolumn{1}{p{0.65\textwidth}}{#6} \\
  % \end{tabular}
}

% Contenu des en-têtes/pied de pages
\fancyhead[LE,RO]{\DocumentProject}
\fancyhead[RE,LO]{\DocumentTitle}
\fancyfoot[CE,CO]{\leftmark}
\fancyfoot[LE,RO]{\thepage}
