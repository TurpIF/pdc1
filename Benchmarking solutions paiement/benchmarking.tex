\documentclass[10pt, a4paper]{article}

% On écrit en français
\usepackage[utf8]{inputenc}
\usepackage[frenchb]{babel}
\usepackage[T1]{fontenc}

% Packages nécessaires
\usepackage{graphicx}
\usepackage{hyperref}

% Document "fancy"
\usepackage{fancyhdr}
\pagestyle{fancy}
\fancyhf{}

% Gros en-têtes/pied de pages
\renewcommand{\headrulewidth}{2pt}
\renewcommand{\footrulewidth}{1pt}

% Police Helvetica <3
\usepackage{helvet}
\renewcommand*{\familydefault}{\sfdefault}

% Enlever les alinéas
\setlength{\parindent}{0pt}

% Sous titre de document
\usepackage{titling}
\newcommand{\subtitle}[1]{%
  \posttitle{%
    \par\end{center}
    \begin{center}\large#1\end{center}
    \vskip0.5em}%
}

% Nom du projet
\def \DocumentProject {Dématérialisation d'un processus de paiement}

% En tête complet de document
\newcommand{\Document}[1]{%
    \def \DocumentTitle {#1}

    \title{\DocumentTitle}
    \subtitle{\DocumentProject}
    \author{
        COMETS Jean-Marie \\
        DELMARRE Adrian \\
        REYNOLDS Nicolas \\
        TURPIN Pierre
    }
    \date{\today}

    \maketitle \newpage

    \tableofcontents \newpage
}

% Contenu des en-têtes/pied de pages
\fancyhead[LE,RO]{\DocumentProject}
\fancyhead[RE,LO]{\DocumentTitle}
\fancyfoot[CE,CO]{\leftmark}
\fancyfoot[LE,RO]{\thepage}


\begin{document}

\Document{Benchmarking des solutions de paiement}

\section{Introduction}
Un certain nombre de standards et de réglementations sont utilisés sur le
marché actuel des titres restaurants : une certification de l’Etat est
indispensable pour rentrer sur le marché, les titres restaurants étant éxonérés
d’impôts. La Commission des Titres Restaurants est l’entité qui s’occupe de
ces régulations. \\

Le marché est actuellement dans une phase de transition avec le passage aux
solutions dématérialisées, dans laquelle s’inscrit le projet. \\

Pour notre étude nous différencions les titres matérialisés des titres
dématérialisés. \\

\section{Titres matérialisés}
Les titres matérialisés sont des supports de paiement remis par l'employeur au
salarié pour lui permettre d'acquitter tout ou partie du prix de son repas
compris dans l'horaire de travail journalier. \\

Ils sont présents en France sous plusieurs formes :
\begin{itemize}
  \item Chèques déjeuner
  \item Tickets restaurants
  \item Chèques de table
  \item Chèques restaurants
\end{itemize}

~\\
En France, la participation de l'employeur est comprise entre 50\% et 60\% de
la valeur du titre. La CTR fixe des frais de traitement. \\

\section{Titres dématérialisés}
En 2012, deux nouveaux acteurs rentrent sur le marché, en proposant une
solution dématérialisée :
\begin{itemize}
  \item Monéo-Resto
  \item Résto-Flash
\end{itemize}

~\\
\textbf{Résto-Flash} fournit des titres restaurant numériques sur téléphone. Il
est le premier acteur à avoir switché sur le dématérialisé. \\

La commission est fixé à 1,75\% sur les transactions effectuées et l'adhésion
mensuelle est à 6,90\euro. \\

\textbf{Moneo Resto} a lancé le concept de carte titres-restaurant
dématérialisée, qui repose sur une carte à puce MasterCard. \\

La carte est rechargeable à distance. Il n'y a donc pas besoin d'employé pour
effectuer de transaction. \\

Les données relatives au solde du compte ne sont pas sur la carte, car une
autorisation est systématiquement demandée au réseau Moneo-Resto. \\

Avantages :
\begin{itemize}
  \item CB (sécurité, pratique, opposition, rechargement non nécessaire)
  \item portefeuille électronique (pas de code, petits montants acceptés)
  \item commissions moins onéreuses
\end{itemize}

\section{Sources}
\url{http://www.carte-ticket-restaurant.fr/Employeurs/faq} \\

\url{http://fr.wikipedia.org/wiki/Titre_restaurant} \\

\url{http://www.moneo-resto.fr/decideurs/entreprise/decouvrez-la-carte-moneo-resto} \\

\url{http://exonerationfiscale.com/titres-services-titres-restaurants} \\

\end{document}

% vim: ft=tex et sw=2 sts=2
