\documentclass[10pt, a4paper]{article}

% On écrit en français
\usepackage[utf8]{inputenc}
\usepackage[frenchb]{babel}
\usepackage[T1]{fontenc}

% Packages nécessaires
\usepackage{graphicx}
\usepackage{hyperref}

% Numérotation de page custom
\usepackage{fancyhdr}
\usepackage{lastpage}
\pagestyle{fancy}
\fancyhf{}
\rfoot{Page \thepage \hspace{1pt} sur \pageref{LastPage}}

% Police Helvetica <3
\usepackage{helvet}
\renewcommand*{\familydefault}{\sfdefault}

% Enlever les alinéas
\setlength{\parindent}{0pt}

% Marges plus larges pour faire moins LaTeX
\usepackage[left=3cm, right=3cm]{geometry}

% Sous titre de document
\usepackage{titling}
\newcommand{\subtitle}[1]{%
  \posttitle{%
    \par\end{center}
    \begin{center}\large#1\end{center}
    \vskip0.5em}%
}

% En tête complet de document
\newcommand{\Document}[1]{%
    \title{#1}
    \subtitle{Dématérialisation d'un processus de paiement}
    \author{
        COMETS Jean-Marie \\
        DELMARRE Adrian \\
        REYNOLDS Nicolas \\
        TURPIN Pierre
    }
    \date{\today}

    \maketitle \newpage

    \tableofcontents \newpage
}


\begin{document}

\Document{Plan d'assurance qualité}

\section{Présentation du plan d'assurance qualité}
\subsection{Présentation générale}
% TODO modifier cette section à chaque nouveau livrable
Ce document représente la version préliminaire du Plan d'Assurance Qualité
(PAQ). Il sera revu et complété au cours de la prestation des services
effectués. \\

En tant que version préliminaire, si des informations de ce document
contredisent des éléments d'un document d'expression du besoin ou de
présentation de solution, c'est ce dernier qui fait foi. \\

Le présent document s'applique aux prestations confiées par Aventix dans le
cadre du projet de conception \textit{"Dématérialisation d'un processus de
paiement"}. \\

\subsection{Objet du document}
Le PAQ est l'engagement de Aventix quant à la politique d'assurance qualité
applicable au projet.
Il rappelle également nos obligations envers Aventix sur le projet de conception.
\\

Le PAQ décrit :
\begin{itemize}
  \item La présentation générale du Plan d'Assurance Qualité.
  \item Les documents à livrer dans le cadre de ce projet de conception.
  \item La méthodologie applicable pour la réalisation des livrables.
  \item Le processus de gestion de la qualité.
  \item Le glossaire des principaux termes, expressions et abbréviations
    métiers utilisées
\end{itemize}

\section{Documents à livrer dans le cadre du projet de conception}
La liste des livrables brièvement décrite est fournie dans le document
d'initialisation du projet. \\

Voici un rappel des documents applicables :
\begin{itemize}
  \item Dossier d'initialisation
  \item Expression des besoins
  \item Architecture applicative
  \item Architecture technique
  \item Business plan
\end{itemize}

\subsection{Dossier d'initialisation}
Le dossier d'initialisation est un document rappelant le contexte et l'objet du
projet de conception suivi. \\

Ce dernier établi également la liste des livrables ainsi qu'une description
schématisée et textuelle de l'organisation de l'équipe dans la gestion des
tâches. \\
Enfin, une analyse de risques simplifiée est faite pour évaluer la cohérence du
projet. \\

\begin{description}
  \item[Contexte et objet du projet] Rappel du contexte du projet, de
    l'entreprise, le milieu dans lequel elle évolue, les concurrents.
    Solutions envisagées pour répondre aux besoins (expression des besoins
    très courte).
  \item[Description des livrables] Chaque livrable avec une description courte
    de son contenu.
  \item[Gestion des tâches] Fonctionnement des tâches et gestion de l'équipe au
    sein du projet.
  \item[Organisation de l'équipe] Listing des membres de l'équipe et de leur
    qualification.
  \item[Analyse des risques] Analyse des risques aux sein du projet (gestion
    projet, livrable, contenu, retard, respect des attentes, respect de
    la charte, du fonctionnement, qualité).
\end{description}

\subsection{Étude comparative (Benchmarking)}
Comparaison de la solution envisagée par rapport aux autres solutions de
paiements et leur interconnexion avec un SI.

\subsection{Expression des besoins}
\begin{enumerate}
  \item Description des attentes du client
  \item Listing point par point des problèmes et contraintes demandés par le
    client. \textbf{important}
  % On dit ce que le client il veut.
  % Ce qu'il veut faire.
  % Ses attentes.
  % Ce qu'il a envisagé de faire
  % La solution bete
  % Ses contraintes.
  % Son budget, le marché existant, concurrence
  \item Description des solutions envisagées.
    \\ Attentes déduites des demandes du client.
    \\ Mise en rapport point par point des attentes avec ce qui va être
    effectué.
\end{enumerate}

\subsection{Architecture Applicative}

% TODO remplir cette section

\subsection{Architecture Technique}
\begin{enumerate}
  \item Stockage
  \item Recherche
  \item Consultation
  \item Communication avec les partenaires et les employés
  \item Sécurité
  \item Disponibilité 24/24 7/7
  \item Création de services métiers facile
  \item SI évolutif et maintenable
  \item (Privilégier les offres cloud computing orienté service SOA)
\end{enumerate}

\subsection{Business plan}
\begin{enumerate}
  \item Produit service
  \item Étude du marché
  \item Plan marketing
  \item Plan d'opération
  \item Stratégie de sortie
  \item Évaluation des couts
  \item Analyse des risques
\end{enumerate}

\section{Méthodologie de réalisation}
\subsection{Réunion de travail}
Afin d'avoir un comportement agile dans l'élaboration des prestations, l'équipe
entière doit se réunir au moins une fois par semaine. Cette réunion permet à
chacun d'échanger son avancée aux autres afin de vérifier et de corriger le
planning prévisionnel établi pour le projet. \\

Cette agilité de travail nous permet d'établir une politique d'organisation
scrum. La gestion de projet se fait donc plus facilement en découpant les
travaux à effectuer en \textit{sprints} dynamiques qui peuvent être modifiés
d'une réunion à une autre afin réagir à une situation de retard, voire d'avance
de la prestation. \\

Un face-à-face hebdomadaire est également prévu avec le prestataire Aventix.
Cela permettra de démontrer notre engagement et la qualité de la prestation
tout en recevant un retour direct pour clarifier l'expression des besoins et
l'établissement d'une solution technique ainsi qu'un business plan. \\

\subsection{Maîtrise des livrables}
\subsubsection{Format des livrables}

\subsubsection{Remise des livrables}
La remise des livrables au client Aventix se fera au fur et à mesure de l'étude
de la conception. \\

Le \textbf{mercredi 15 octobre 2014}, le dossier d'initialisation sera transmis.
Une étude comparative sur les lecteurs de cartes à puce et leur interconnexion
avec un système d'information sera également donnée pour établir et justifier
notre choix de conception sur les architectures applicative et technique
choisies. \\
L'expression des besoins sera également fournis. \\

Le \textbf{mercredi 22 octobre 2014}, le dossier d'architecture applicative sera rendu
et validé. \\

Le \textbf{mercredi 5 novembre 2014}, le dossier d'architecture technique sera délivré.
Un document expliquant la solution EDI choisie pour l'échange avec la banque
sera également remis. \\

Le business plan sera donné au plus tard le \textbf{mercredi 19 novembre 2014}. \\

Une revue finale avec Aventix sera dispensée entre le \textbf{mercredi 5
  novembre} et le \textbf{mercredi 19 novembre} 2014 sur rendez-vous et accord
des deux parties.  Cette réunion d'explication nous permettra d'expliquer la
démarche effectué et la solution choisie durant la prestation afin d'améliorer
la prise en main du business plan et d'assurer la qualité des directives
métiers de la part d'Aventix. \\

\subsection{Maîtrise des risques}
Les méthodologies assure une bonne maîtrise des différents risques possible
durant l'étude du projet de conception. \\

En effet, l'organisation de travail agile permet de correctement prévenir les
risques d'écart du planning prévisionnel. Comme ce dernier est dynamique et
peut être ajusté à chaque itération, le rythme de travail peut être modifié
afin de s'en rapprocher le plus. Cela est valable pour les retards mais
également pour les surplus de travail. \\

Les revues et validations régulières de la part d'Aventix permettent d'avoir un
retour d'approbation et de contentement, limitant les risques de non
satisfaction de la prestation fournie. \\

Enfin, le simple suivi des règles de gestion de la qualité de ce document, qui
sont approuvées par Aventix, empêchent tomber dans le cas du paragraphe
précédent. \\

\section{Gestion de la qualité}
La gestion de la qualité de la prestation est assurée par un suivi systématique
de ce document ainsi que grâce à la méthodologie de travail de l'équipe. \\

De plus, le processus d'écriture des livrables requiert la relecture par le
responsable qualité de l'équipe. Cette phase de correction permet, en plus
d'avoir un document propre, de gérer correctement la cohérence entre les
différents documents. \\

Les termes métiers utilisés dans les documents sont alors répertoriés dans le
glossaire et unifiés à travers tous les documents. \\
Également les documents doivent remplir les conditions de qualité attendu
spécifié dans ce document. \\

Dans le cas d'un non-respect grave des attentes, le document est redistribué à
l'auteur pour amélioration pour repasser ensuite par une phase de relecture et
correction. \\
Sinon, le document est renvoyé à l'auteur afin qu'il approuve les modifications
pour valider définitivement le document. \\

Dans le cas où un travail requiert des attentes techniques spécifiques, un de
nos expert peut faire partie du processus de correction afin de certifier la
qualiter des livrables finaux. \\

\section{Glossaire}

% TODO remplir cette section

\end{document}

% vim: ft=tex et sw=2 sts=2
