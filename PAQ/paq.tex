\documentclass[11pt, a4paper]{article}

\usepackage[utf8]{inputenc}
\usepackage{amsfonts}
\usepackage{amsmath}
\usepackage{hyperref}
\usepackage{graphicx}
\usepackage[frenchb]{babel}
\usepackage[left=3cm, right=3cm]{geometry}

% Headers & footers
\usepackage{fancyhdr}
\pagestyle{fancyplain}
\fancyhead[LE]{\fancyplain{}{}}
\fancyhead[CE]{\fancyplain{}{}}
\fancyhead[RE]{\fancyplain{}{\bfseries\leftmark}}
\fancyhead[LO]{\fancyplain{}{\bfseries\rightmark}}
\fancyhead[CO]{\fancyplain{}{}}
\fancyhead[RO]{\fancyplain{}{}}
\fancyfoot[LE]{\fancyplain{}{}}
\fancyfoot[CE]{\fancyplain{}{}}
\fancyfoot[RE]{\fancyplain{}{}}
\fancyfoot[LO]{\fancyplain{}{}}
\fancyfoot[CO]{\fancyplain{}{\bfseries\thepage}}
\fancyfoot[RO]{\fancyplain{}{}}
% \renewcommand{\footrulewidth}{0.4pt}

\begin{document}

\title{Plan d'assurance qualité}
\author{
  COMETS Jean-Marie \\
  DELMARRE Adrian \\
  REYNOLDS Nicolas \\
  TURPIN Pierre
}
\date{\today}

\maketitle \newpage

% \tableofcontents \newpage

% Introduction (5 pages)
\section{Dossier d'initialisation}
Plan :
  \\ Contexte et objet du projet
    \\ Rappeler le contexte du projet, de l'entreprise. Le milieu dans lequel elle
    évolue, les concurrents.
    \\ On montre les solutions envisagés pour répondre aux besoin (expression des
    besoins très courte).
  \\ Description des livrables
    \\ Chaque livrable avec une description courte de son contenu
  \\ Gestion des tâches
    \\ Fonctionnement des tâches et gestion de l'équipe au sein du projet.
  \\ Organisation de l'équipe
    \\ Listing des membres de l'équipe et de leur qualification.
  \\ Analyse des risques
    \\ Analyse des risques aux sein du projet (qu'est-ce qu'il y a comme risque :
    gestion projet, livrable, contenu, retard, respect des attentes, respect de
    la charte, du fonctionnement, qualité, tout ça quoi).

\section{Expression des besoins}
Description des attentes du client
\\ Listing point par point des problèmes et contraintes demandés par le client.
\textbf{important}
\\ On dit ce que le client il veut.
\\ Ce qu'il veut faire.
\\ Ses attentes.
  \\ Ce qu'il a envisagé de faire
  \\ La solution bete
\\ Ses contraintes.
  \\ Son budget, le marché existant, concurrence
\\ Description de ce qu'on va faire.
  \\ Description de ce qui est compris et ce qu'on déduit de ses attentes.
  \\ Description point par point comparé à ce qu'on va réellement faire.

\section{Description des solutions}
Architecture applicative
\\ Architecture technique
  \\ Stockage
  \\ Recherche
  \\ Consultation
  \\ Communication avec les partenaires et les employés
  \\ Sécurité
  \\ Disponibilité 24/24 7/7
  \\ Création de services métiers facile
  \\ SI évolutif et maintenable
  \\ (Priviligier les offres cloud computing orienté service SOA)
\\ Benchmarking
  \\ Comparaison de la solution envisagé par rapport aux autres solutions de
  \\ paiements et leur interconnexion avec un SI.

\section{Business plan}
Produit service
Etude du marché
Plan marketing
Plan d'opération
Stratégie de sortie
Evaluation des couts
Analyse des risques

\section{Glossaire}

\end{document}
