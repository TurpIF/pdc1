\documentclass[10pt, a4paper]{article}

% On écrit en français
\usepackage[utf8]{inputenc}
\usepackage[frenchb]{babel}
\usepackage[T1]{fontenc}

% Packages nécessaires
\usepackage{graphicx}
\usepackage{hyperref}

% Numérotation de page custom
\usepackage{fancyhdr}
\usepackage{lastpage}
\pagestyle{fancy}
\fancyhf{}
\rfoot{Page \thepage \hspace{1pt} sur \pageref{LastPage}}

% Police Helvetica <3
\usepackage{helvet}
\renewcommand*{\familydefault}{\sfdefault}

% Enlever les alinéas
\setlength{\parindent}{0pt}

% Marges plus larges pour faire moins LaTeX
\usepackage[left=3cm, right=3cm]{geometry}

% Sous titre de document
\usepackage{titling}
\newcommand{\subtitle}[1]{%
  \posttitle{%
    \par\end{center}
    \begin{center}\large#1\end{center}
    \vskip0.5em}%
}

% En tête complet de document
\newcommand{\Document}[1]{%
    \title{#1}
    \subtitle{Dématérialisation d'un processus de paiement}
    \author{
        COMETS Jean-Marie \\
        DELMARRE Adrian \\
        REYNOLDS Nicolas \\
        TURPIN Pierre
    }
    \date{\today}

    \maketitle \newpage

    \tableofcontents \newpage
}

\usepackage{float}
\usepackage[default]{frcursive}
\usepackage[T1]{fontenc}

\begin{document}

\Document{Besoins en infrastructure}

\section{Prologue}

\cursive{
  \itshape
  Le présent document, malgré tous nos efforts, \\
  Ignorera les us, la forme et quelques autres,  \\
  Ordinairement lus dans ce cadre retors. \\
  N'y voyez, par pitié, d'embarras que le nôtre, \\
  Et abstrayez-vous donc de tous vos préjugés. \\
  La lecture, pour vous, en sera plus aisée. \\

  Bien. La cause entendue, il nous faut commencer. \\
  Rien, dans ce beau dossier de mille pages crues, \\
  Un peu moins, un peu plus, à quelques lignes près, \\
  Ne viendra perturber votre lecture émue. \\
  Il s'agit des besoins en lieu d'infrastructure, \\
  Et des calculs, aussi, de notre architecture. \\
}

\section{Dimensions et mesures}
\paragraph{Partenaires}
~\\
\cursive{
  \itshape
  Notre ambitieux projet ne se veut solitaire. \\
  On espère compter de nombreux partenaires. \\
  Un décompte avancé des hypers de banlieue \\
  En trouve cinq milliers, prospères et heureux. \\

  Douze milliers de bornes, en leur sein disposées, \\
  Par six millions d'esprits seront sollicitées. \\
  Ce devrait soulager, aux heures des repas, \\
  Ces grands supermarchés des files de cabas. \\

  Dans la restauration, nous viserons plus grand : \\
  D'abord cinqante mille, et plus avec le temps. \\
  Ce sont donc, au final, soixante-deux milliers \\

  De terminaux banals qui seront dispersés. \\
  Chaque jour que Dieu fait, on attend deux millions \\
  Et quatre cent milliers de belles transactions. \\
}
\paragraph{Transactions}
~\\
\cursive{
  \itshape
  De ces transactions, nous pouvons deviser : \\
  Faites attention, cela va se corser, \\
  Puisque l'information devant être portée \\
  Contraint la solution que l'on va proposer. \\

  Notre réflexion est pour moitité fondée \\
  Sur la contraction de certaines monnaies : \\
  Le Bitcoin et l'Ether imposent de stocker \\
  L'ensemble des transferts qui un jour ont été. \\

  Chaque occurrence, alors, pèsera, esseulée, \\
  Moins de trois mille octets, pour limiter l'effort \\
  Et la charge, au plus fort, qu'on pourrait demander. \\

  Enfin, pour archiver, et stocker le rapport \\
  Sans être débordés, on a trouvé pratique \\
  De tout optimiser d'un air logarithmique. \\
}
\paragraph{Débits}
~\\
\cursive{
  \itshape
  Mais nos trois mille octets font une belle somme. \\
  Au cœur de la journée, en quelques vingt minutes, \\
  Plus de sept cent milliers d'acheteurs autonomes \\
  Ne sauront tolérer que notre art ne se butte. \\

  Ce sont donc, à peu près, six cent transactions \\
  Par seconde passée au plus fort de l'action. \\
  Et l'on peut calculer, sans complication, \\
  Le débit demandé, par multiplication. \\

  Pour les transactions, le produit calculé \\
  Amène au million, et huit cent bons milliers \\
  D'octets chaque seconde, au bas mot, sans forcer. \\

  Prenons en compte aussi les différents accès \\
  Des comptables assis, les dépôts, les retraits, \\
  Qui décuplent la ronde au plus fort de l'année. \\
}

\section{Dispositions, mesures}
\paragraph{Architecture}
~\\
\cursive{
  \itshape
  Nos besoins en débit restant plutôt modestes, \\
  Considérant aussi, noyé dans tout le reste, \\
  Le pendant « qualité » qu'il nous faut observer, \\
  Nous avons décidé de tout distribuer. \\

  Nous aurons, au début, deux modestes serveurs, \\
  Sur lesquels, virtuels, se verront répliqués \\
  D'autres serveurs, en sus, qui seront le moteur \\
  De notre rationnel plan de surqualité. \\

  Dupliqués en nuage, et gardant bien cachée \\
  Une part, un étage, un pendant des données \\
  De la base éponyme, et qu'on garde en local, \\

  Ces serveurs, verbatim, devront pouvoir, sans mal, \\
  Esseulés, sans délai, toute charge assumer. \\
  En local, les données sont aussi dupliquées. \\
}
\paragraph{Cumulus}
~\\
\cursive{
  \itshape
  Nos besoins en débit étant bien définis, \\
  Il nous reste, méshui, à préciser aussi \\
  Nos besoins en stockage, en calcul, en mémoire, \\
  Sur la part où s'engage un prestataire avare. \\

  Comme nous avons pris le parti, assumé, \\
  De virtualiser les serveurs Web, ainsi \\
  Que la sécurité (firewall et proxy), \\
  Pour la RAM aujourd'hui, seize gigas iraient. \\

  En calculs, nos besoins (en lieu de processeur), \\
  Sont légers : on est bien avec juste huit cœurs \\
  Pour paralléliser les serveurs virtuels. \\

  Le stockage, il est vrai, est un peu exigeant : \\
  Il veut des SSD en miroir (grossissant). \\
  Deux d'entre eux suffiraient, de cent gigas réels. \\
}
\paragraph{Domus}
~\\
\cursive{
  \itshape
  Notre cœur de métier étant le traitement \\
  Des données des clients que l'on peut exploiter \\
  Il nous faut disposer notre base, pourtant, \\
  Sur des serveurs puissants et faciles d'accès. \\

  Les données structurées demandent de la place \\
  Et quand même archivées, elles restent gourmandes, \\
  Les calculs associés, en mémoire, voraces, \\
  Contraignent notre baie à de fortes demandes. \\

  En outre, il faut penser que malgré nos efforts, \\
  Pour stocker les données, il faudra du renfort, \\
  Au bout de quelques mois, ou bien quelques années. \\

  Il faut déjà prévoir que notre architecture \\
  Sans soufrir de surseoir, et nos infrastructures \\
  Devront porter ce poids, et vite évoluer. \\
}

\section{Épilogue}
\cursive{
  \itshape
  Après tant d'attention, vos efforts vont payer \\
  Moyennant un dernier moment de compassion. \\
  Grand repos, et après, enfin la rémission, \\
  Heureuse punition pour avoir toléré \\
  Avecques indulgence, en lieu de scepticisme, \\
  Ratages, négligence, vous êtes magnanime. \\
  
  Comme un train de poussière ignorant où il va \\
  Ou sur quelle misère au coin du jour il pend \\
  Modeste pour le moins, ce cahier va cesser \\
  Et nous mettrons enfin un grand coup de balai \\
  Taisant pour un instant qui, on croit, durera \\
  Sa musique, son chant, qui ravissaient antan. \\
 }

\end{document}
