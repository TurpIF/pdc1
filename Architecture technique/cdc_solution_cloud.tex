\documentclass[10pt, a4paper]{article}

% On écrit en français
\usepackage[utf8]{inputenc}
\usepackage[frenchb]{babel}
\usepackage[T1]{fontenc}

% Packages nécessaires
\usepackage{graphicx}
\usepackage{hyperref}

% Document "fancy"
\usepackage{fancyhdr}
\pagestyle{fancy}
\fancyhf{}

% Gros en-têtes/pied de pages
\renewcommand{\headrulewidth}{2pt}
\renewcommand{\footrulewidth}{1pt}

% Police Helvetica <3
\usepackage{helvet}
\renewcommand*{\familydefault}{\sfdefault}

% Enlever les alinéas
\setlength{\parindent}{0pt}

% Sous titre de document
\usepackage{titling}
\newcommand{\subtitle}[1]{%
  \posttitle{%
    \par\end{center}
    \begin{center}\large#1\end{center}
    \vskip0.5em}%
}

% Nom du projet
\def \DocumentProject {Dématérialisation d'un processus de paiement}

% En tête complet de document
\newcommand{\Document}[1]{%
    \def \DocumentTitle {#1}

    \title{\DocumentTitle}
    \subtitle{\DocumentProject}
    \author{
        COMETS Jean-Marie \\
        DELMARRE Adrian \\
        REYNOLDS Nicolas \\
        TURPIN Pierre
    }
    \date{\today}

    \maketitle \newpage

    \tableofcontents \newpage
}

% Contenu des en-têtes/pied de pages
\fancyhead[LE,RO]{\DocumentProject}
\fancyhead[RE,LO]{\DocumentTitle}
\fancyfoot[CE,CO]{\leftmark}
\fancyfoot[LE,RO]{\thepage}

\usepackage{float}
\usepackage[default]{frcursive}
\usepackage[T1]{fontenc}

\begin{document}

\Document{Besoins en infrastructure}

\section{Prologue}

\cursive{
  \itshape
  Le présent document, malgré tous nos efforts, \\
  Ignorera les us, la forme et quelques autres,  \\
  Ordinairement lus dans ce cadre retors. \\
  N'y voyez, par pitié, d'embarras que le nôtre, \\
  Et abstrayez-vous donc de tous vos préjugés. \\
  La lecture, pour vous, en sera plus aisée. \\

  Bien. La cause entendue, il nous faut commencer. \\
  Rien, dans ce beau dossier de mille pages crues, \\
  Un peu moins, un peu plus, à quelques lignes près, \\
  Ne viendra perturber votre lecture émue. \\
  Il s'agit des besoins en lieu d'infrastructure, \\
  Et des calculs, aussi, de notre architecture. \\
}

\section{Dimensions et mesures}
\paragraph{Partenaires}
~\\
\cursive{
  \itshape
  Notre ambitieux projet ne se veut solitaire. \\
  On espère compter de nombreux partenaires. \\
  Un décompte avancé des hypers de banlieue \\
  En trouve cinq milliers, prospères et heureux. \\

  Douze milliers de bornes, en leur sein disposés, \\
  Par six millions d'esprits seront sollicitées. \\
  Ce devrait soulager, aux heures des repas, \\
  Ces grands supermarchés des files de cabas. \\

  Dans la restauration, nous viserons plus grand : \\
  D'abord cinqante mille, et plus avec le temps. \\
  Ce sont donc, au final, soixante-deux milliers \\

  De terminaux banals qui seront dispersés. \\
  Chaque jour que Dieu fait, on attend deux millions \\
  Et quatre cent milliers de belles transactions. \\
}
\paragraph{Transactions}
~\\
\cursive{
  \itshape
  De ces transactions, nous pouvons deviser : \\
  Faites attention, cela va se corser, \\
  Puisque l'information devant être portée \\
  Contraint la solution que l'on va proposer. \\

  Notre réflexion est pour moitité fondée \\
  Sur la contraction de certaines monnaies : \\
  Le Bitcoin et l'Ether imposent de stocker \\
  L'ensemble des transferts qui un jour ont été. \\

  Chaque occurrence, alors, pèsera, esseulée, \\
  Moins de trois mille octets, pour limiter l'effort \\
  Et la charge, au plus fort, qu'on pourrait demander. \\

  Enfin, pour archiver, et stocker le rapport \\
  Sans être débordés, on a trouvé logique \\
  De tout optimiser d'un air logarithmique. \\
}
\paragraph{Débits}
~\\
\cursive{
  \itshape
  \\
  \\
  \\
  \\

  \\
  \\
  \\
  \\

  \\
  \\
  \\

  \\
  \\
  \\
}

\section{Dispositions, mesures}
\paragraph{Architecture}
~\\
\cursive{
  \itshape
  \\
  \\
  \\
  \\

  \\
  \\
  \\
  \\

  \\
  \\
  \\

  \\
  \\
  \\
}
\paragraph{Amazonus}
~\\
\cursive{
  \itshape
  \\
  \\
  \\
  \\

  \\
  \\
  \\
  \\

  \\
  \\
  \\

  \\
  \\
  \\
}
\paragraph{Domus}
~\\
\cursive{
  \itshape
  \\
  \\
  \\
  \\

  \\
  \\
  \\
  \\

  \\
  \\
  \\

  \\
  \\
  \\
}

\section{Epilogue}
\cursive{
  \itshape
  B \\
  U \\
  E \\
  N \\
  A \\
  S \\

  N \\
  O \\
  C \\
  H \\
  E \\
  S \\
  }

\end{document}
