\documentclass[10pt, a4paper]{article}

% On écrit en français
\usepackage[utf8]{inputenc}
\usepackage[frenchb]{babel}
\usepackage[T1]{fontenc}

% Packages nécessaires
\usepackage{graphicx}
\usepackage{hyperref}

% Document "fancy"
\usepackage{fancyhdr}
\pagestyle{fancy}
\fancyhf{}

% Gros en-têtes/pied de pages
\renewcommand{\headrulewidth}{2pt}
\renewcommand{\footrulewidth}{1pt}

% Police Helvetica <3
\usepackage{helvet}
\renewcommand*{\familydefault}{\sfdefault}

% Enlever les alinéas
\setlength{\parindent}{0pt}

% Sous titre de document
\usepackage{titling}
\newcommand{\subtitle}[1]{%
  \posttitle{%
    \par\end{center}
    \begin{center}\large#1\end{center}
    \vskip0.5em}%
}

% Nom du projet
\def \DocumentProject {Dématérialisation d'un processus de paiement}

% En tête complet de document
\newcommand{\Document}[1]{%
    \def \DocumentTitle {#1}

    \title{\DocumentTitle}
    \subtitle{\DocumentProject}
    \author{
        COMETS Jean-Marie \\
        DELMARRE Adrian \\
        REYNOLDS Nicolas \\
        TURPIN Pierre
    }
    \date{\today}

    \maketitle \newpage

    \tableofcontents \newpage
}

% Contenu des en-têtes/pied de pages
\fancyhead[LE,RO]{\DocumentProject}
\fancyhead[RE,LO]{\DocumentTitle}
\fancyfoot[CE,CO]{\leftmark}
\fancyfoot[LE,RO]{\thepage}


\begin{document}

\Document{Architecture technique (JM)}

\section{Dialogue bornes}

Le dialogue avec les bornes peut être considérée comme l'activité la plus
coûteuse, en visant un quart du marché des titres restaurants, soit 1M
d'utilisateurs, on peut estimer la charge maximale à 80k connexions
simultanées (80\% des utilisateurs le midi au maximum, répartis par tranches de
10\%). \\

Les bornes sont reliées à un serveur de dialogue bornes $B$, s'occupant
d'effectuer les opérations spécifiques aux bornes (transaction, état, etc...).
Ce serveur devant gérer une charge conséquente, on lui appliquera une
\textbf{balance de charge}, distribuant entre deux autres serveurs équivalents.
Chacun de ces serveurs sera installé sur une instance séparée. \\

Pour la solution utilisée pour le dialogue, on implémentera une solution
spécifique, au vu de la spécificité du problème. La communication sera sur la
base du TCP, pour garantir la fiabilité du transport. \\

\subsection{Comparaison des solutions de balance de charge}

On pourra utiliser \textit{Nginx} pour la balance de charge, qui est une
solution libre, performante et stable. \\

Une solution concurrente est \textit{Apache}, mais à l'inconvénient d'être
lourde et moins rapide au vu des dernières comparaisons.

\section{BDD}

On utilisera une solution de base de données libre car peu couteuse, en
préférant une solution relationnelle pour profiter de la conception préalable
et déduire la structure interne. \\

Les trois serveurs de dialogue bornes auront chacun une réplication de la base
de données, pour limiter davantage les accés réseau et ainsi diminuer la
latence de réponse aux bornes en période de charge conséquente. \\

L'accés à la base de données \textbf{maître}, centralisant les réplications,
sera limitée aux différentes machines comprenant des applications Aventix, via
des règles firewall. L'accés aux réplications de base de données sera
entièrement fermé aux connexions externes.

\subsection{Comparaison des solutions de base de données libres}

Deux solutions concurrentes se détachent :

\begin{description}
  \item[MySQL] actuellement maintenu par \textbf{Oracle}, c'est une solution
    ayant l'avantage d'être relativement simple à mettre en place.
  \item[PostgreSQL] toujours un projet 100\% libre, c'est une solution plus
    complèxe à mettre en place, mais les dernières comparaisons la mettent en
    pole position au niveau de la performance au niveau de la distribution de
    requêtes (architecture maître-esclave, soit celle-ci).
\end{description}

\section{Gestion utilisateur / entreprise / commerçant}

Chaque application de gestion sera installée sur une instance séparée, et ne
devra pas supporter une charge trop conséquente. Le serveur utilisé devra gérer
des connexions HTTP régulières, on choisira parmi les solutions libres encore
une fois, pour réduire le cout de la solution. Le développement de l'interface
demandera cependant un investissement non négligeable, vu que c'est la
visibilité d'Aventix qui est en jeu. \\

L'accés à ces applications sera ouvert en HTTP pour l'intégralité d'internet,
vu que l'accés sera global. On considérera toutefois la limitation de l'accés à
l'application de gestion commerçant dans le cas d'un contrat conséquent avec un
unique commerçant.

\section{Gestion SI}

L'application Aventix sera installée sur une instance séparée, dont l'accés
sera limité par un firewall plus restrictif, spécifiant de n'accepter que les
connexions spécifiques au réseau d'Aventix.

\end{document}
