\documentclass[10pt, a4paper]{article}

% On écrit en français
\usepackage[utf8]{inputenc}
\usepackage[frenchb]{babel}
\usepackage[T1]{fontenc}

% Packages nécessaires
\usepackage{graphicx}
\usepackage{hyperref}

% Document "fancy"
\usepackage{fancyhdr}
\pagestyle{fancy}
\fancyhf{}

% Gros en-têtes/pied de pages
\renewcommand{\headrulewidth}{2pt}
\renewcommand{\footrulewidth}{1pt}

% Police Helvetica <3
\usepackage{helvet}
\renewcommand*{\familydefault}{\sfdefault}

% Enlever les alinéas
\setlength{\parindent}{0pt}

% Sous titre de document
\usepackage{titling}
\newcommand{\subtitle}[1]{%
  \posttitle{%
    \par\end{center}
    \begin{center}\large#1\end{center}
    \vskip0.5em}%
}

% Nom du projet
\def \DocumentProject {Dématérialisation d'un processus de paiement}

% En tête complet de document
\newcommand{\Document}[1]{%
    \def \DocumentTitle {#1}

    \title{\DocumentTitle}
    \subtitle{\DocumentProject}
    \author{
        COMETS Jean-Marie \\
        DELMARRE Adrian \\
        REYNOLDS Nicolas \\
        TURPIN Pierre
    }
    \date{\today}

    \maketitle \newpage

    \tableofcontents \newpage
}

% Contenu des en-têtes/pied de pages
\fancyhead[LE,RO]{\DocumentProject}
\fancyhead[RE,LO]{\DocumentTitle}
\fancyfoot[CE,CO]{\leftmark}
\fancyfoot[LE,RO]{\thepage}


\begin{document}

\Document{Architecture technique}

\section{Présentation générale}

La figure \ref{fig:network} détaille l'architecture technique choisie.
Cependant, certains points doivent être justifiés ou davantage expliqués.

\begin{figure}[htpb]
    \centering
    \includegraphics[width=\textwidth]{network}
    \caption{Schéma général de l'architecture technique choisie}
    \label{fig:network}
\end{figure}

\paragraph{Serveurs applicatifs}

L'accès aux serveurs applicatifs est gouverné par une couche \textbf{firewall}
et une couche \textbf{proxy}. La couche firewall est nécessaire pour gérer le
trafic indésirable (se référer à la section
\ref{subsec:securite-trafic} pour plus de détails). La couche proxy
permet de gérer le passage à l'échelle des serveurs applicatifs. \\

Le principe général du passage à l'échelle des serveurs applicatifs est basé
sur la duplication et synchronisation de plusieurs instances des serveurs
applicatifs, avec balance de charge sur ces dernières, régie par le proxy (se
référer à la section {\huge \textbf{TODO} } pour plus de détails).

\paragraph{Base de données}

L'accès au sous-système de base de données est régi par un firewall,
spécifiquement conçu pour le SGBD choisi. L'idée est de bloquer l'accès au
sous-système de base de données au monde extérieur, autorisant uniquement
l'accès au proxy servant à répartir les accès aux différentes base de données
"maîtres". \\

De plus, cache est configuré sur les serveurs de bases de données, pour réduire
la latence due à l'accès au sous-système de base de données, ainsi que de
réduire la charge qui lui est soumise.

\section{Choix de la solution cloud}
\label{sec:choix-solution-cloud}
\section{Passage à l'échelle (scaling)}
\label{sec:scaling}

\section{Sécurité}
\label{sec:securite}

\subsection{Sécurité d'infrastructure}
\label{subsec:securite-infrastructure}

En choisissant une solution cloud, la disponibilité de l'infrastructure est
garantie par le prestataire cloud, en l'occurence \textbf{Amazon AWS}. Le
système peut donc être considéré relativement sécurisé vis-à-vis des attaques
par déni de service (DoS simple), ou autre attaque d'infrastructure. \\

De plus, la disponibilité du système est dépendante de la disponibilité d'AWS,
en l'occurence celle-ci peut être assurée selon le prix de la prestation. Le
taux de panne de 0\% ne peut malheureusement pas l'être, du fait du nombre de
facteurs externes entrant en jeu. Cependant, un taux de 99\%, voire jusqu'à
99.95\% peut l'être par AWS (source: \url{http://aws.amazon.com/ec2/sla/}). \\

En passant par une solution cloud, le système est aussi protégé des attaques
physiques (coupure générale, attaque électromagnétique, etc...), mais encore
dépendant de l'infrastructure d'AWS. \\

Toutefois, une exception aux propositions demeure : le \textbf{sous-système de
base de données ne réside pas intégralement dans le cloud}. Ce problème n'est
pas d'une ampleur catastrophique, il faut noter qu'on est relativement bien
protégé des attaques DoS grâce au pare-feu conçu spécifiquement pour contrôler
l'accès et donc empêcher le trafic intempestif. \\

Malheureusement les attaques physiques peuvent atteindre le sous-système de
base de données, c'est la faiblesse majeure du système. Pour limiter davantage
la vulnérabilité physique du système, deux baies de serveurs en syncronisation
multi-master seront installées sur deux locaux différents (détaillé dans la
section {\huge \textbf{TODO}}.

\subsection{Sécurité du trafic}
\label{subsec:securite-trafic}

Le trafic indésirable correspond au trafic qui n'est pas directement lié à
l'utilisation normale du système. Il peut être utilisé comme attaque visant à 
exploiter des failles d'autres services présents sur la VM, ou tout
simplement à réduire la disponibilité du système en multipliant les
accès (DoS). \\

EC2 met à disposition un firewall pour ses instances, ce qui permet de régler
son accès. Cependant, les VM étant installées à l'intérieur d'une instance,
elles doivent toutes être configurées séparément pour accepter uniquement le
trafic qui les concerne.

\paragraph{Fermeture maximale}

Un document relatant des conseils de sécurisation d'instance EC2, produit par
ce même service, est disponible à l'adresse suivante :
\url{http://aws.amazon.com/articles/1233/} (en date du \today). L'idée est
simplement de n'autoriser que le trafic qui est attendu, et par défaut de
bloquer toute connexion entrante ne correspondant pas à une règle spécifiée.

\paragraph{Chiffrement des messages}

La totalité des échanges de messages avec le système sera effectuée en
utilisant une authentification par clé. Il faudra par exemple acheter un
\textbf{certificat SSL} pour permettre l'utilisation du protocole HTTPS pour
accéder aux différentes applications web. Le dialogue avec les bornes sera
quant à lui chiffré par une méthode utilisant la cryptographie asymétrique (clé
publique).

\end{document}
